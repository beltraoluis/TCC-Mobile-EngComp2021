% REFERÊNCIAS------------------------------------------------------------------
\usepackage[%
    alf,
    abnt-emphasize=bf,
    bibjustif,
    recuo=0cm,
    abnt-url-package=url,       % Utiliza o pacote url
    abnt-refinfo=yes,           % Utiliza o estilo bibliográfico abnt-refinfo
    abnt-etal-cite=3,
    abnt-etal-list=3,
    abnt-thesis-year=final
]{abntex2cite}                  % Configura as citações bibliográficas conforme a norma ABNT

% PACOTES----------------------------------------------------------------------
\usepackage[final]{pdfpages}
\usepackage{booktabs}                                       % Réguas horizontais em tabelas
\usepackage{color, colortbl}                                % Controle das cores
\usepackage{float}                                          % Necessário para tabelas/figuras em ambiente multi-colunas
\usepackage{graphicx}                                       % Inclusão de gráficos e figuras
\usepackage{icomma}                                         % Uso de vírgulas em expressões matemáticas
\usepackage{indentfirst}                                    % Indenta o primeiro parágrafo de cada seção
\usepackage{microtype}                                      % Melhora a justificação do documento
\usepackage{multirow, array}                                % Permite tabelas com múltiplas linhas e colunas
\usepackage{subeqnarray}                                    % Permite subnumeração de equações
\usepackage{lastpage}                                       % Para encontrar última página do documento
\usepackage{verbatim}                                       % Permite apresentar texto tal como escrito no documento, ainda que sejam comandos Latex
\usepackage{amsfonts, amssymb, amsmath}                     % Fontes e símbolos matemáticos
\usepackage[algoruled, portuguese]{algorithm2e}             % Permite escrever algoritmos em português
\usepackage[scaled]{helvet}                                % Usa a fonte Helvetica
%\usepackage{times}                                          % Usa a fonte Times
%\usepackage{palatino}                                      % Usa a fonte Palatino
%\usepackage{lmodern}                                       % Usa a fonte Latin Modern
\usepackage[bottom]{footmisc}                               % Mantém as notas de rodapé sempre na mesma posição
\usepackage{ae, aecompl}                                    % Fontes de alta qualidade
\usepackage{latexsym}                                       % Símbolos matemáticos
\usepackage{lscape}                                         % Permite páginas em modo "paisagem"
%\usepackage{picinpar}                                      % Dispor imagens em parágrafos
%\usepackage{scalefnt}                                      % Permite redimensionar tamanho da fonte
\usepackage{subfig}                                        % Posicionamento de figuras
%\usepackage{upgreek}                                       % Fonte letras gregas
%pra codigo python
\usepackage{listings}
\usepackage{pdfpages}
\usepackage{lipsum}
\usepackage[utf8]{inputenc}
\usepackage[brazil]{babel}
\usepackage[T1]{fontenc}
\usepackage{lmodern}
\usepackage{easylist} %para aninhar enumerate
\usepackage{paralist} %para compactenum



\lstset{language=C,
keywords={do, break,case,catch,continue,else,elseif,end,for,function,global,if,otherwise,persistent,return,switch,try,while,int, float, char},
basicstyle = \ttfamily, % \footnotesize, % Tamanho da fonte do código
numbers = left, % Posição da numeração das linhas
numberstyle = \tiny\color{blue}, % Estilo da numeração de linhas
stepnumber = 1, % Numeração das linhas ocorre a cada quantas linhas?
numbersep = 10pt, % Distância entre a numeração das linhas e o código
backgroundcolor = \color{white}, % Cor de fundo
showspaces = false, % Exibe espaços com um sublinhado
showstringspaces = false, % Sublinha espaços em Strings
showtabs = false, % Exibe tabulação com um sublinhado
%frame = single, % Envolve o código com uma moldura, pode ser single ou trBL
%rulecolor = \color{black}, % Cor da moldura
tabsize = 2, % Configura tabulação em x espaços
captionpos = b, % Posição do título pode ser t (top) ou b (bottom)
breaklines = true, % Configura quebra de linha automática
breakatwhitespace= false, % Configura quebra de linha
title = \lstname, % Exibe o nome do arquivo incluido
%caption = \lstname, % Também é possível usar caption no lugar de title
keywordstyle = \color{blue}, % Estilo das palavras chaves
commentstyle = \color{gray}, % Estilo dos Comentários
}

% Seleção de código de fonte
% Redefine a fonte para uma fonte similar a Arial (fonte Helvetica)
%\renewcommand*\familydefault{\sfdefault}
