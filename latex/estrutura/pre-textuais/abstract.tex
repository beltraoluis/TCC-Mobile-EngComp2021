% ABSTRACT--------------------------------------------------------------------------------

\begin{resumo}[ABSTRACT]
\begin{SingleSpacing}

% Não altere esta seção do texto--------------------------------------------------------
\imprimirautorcitacao. \imprimirtitleabstract. \imprimirdata. \pageref {LastPage} f. \imprimirprojeto\ – \imprimirprograma, \imprimirinstituicao. \imprimirlocal, \imprimirdata.\\
%---------------------------------------------------------------------------------------

With the advancement of processing capacity and the miniaturization of modern processors, with each passing year it's noticeable how much popular the mobile devices are becoming in Brazil. This devices uses embedded systems, where the consumption of resources like batery, processement power and memory, which are scarce, have a vital importance in order to provide a good experience to the user. Currently, mobile devices are a great source of data generation. This data are usually located at the edges of the network, and it is often unfeasible to try to centralize. The fog computing layer was created as a means of providing storage and communication services between cloud computing and the end devices (IoT - Internet of Things), being able to provide real-time data in a decentralized and scalable way. An element called message broker is responsible for managing and distributing information through one or more communication channels, in order to demand less computational resources with repeated requests for a single information. This paper will aim to analyze some communication protocols for fog computing, such as MQTT (Message QueuingTelemetry Transport), AMQP (Advanced Message Queuing Protocol) e STOMP (SimpleText Orientated Messaging Protocol) through the Broker RabbitMQ, in order to analyze how these protocols behave in this challenging new world of scarce resources, which are self-managed by Internal tools of the Operating System.\par

\textbf{Keywords}: Fog computing. Android. Smartphone, AMQP, STOMP, MQTT. 
% Escolha de 3 a 5 palavras ou termos que descrevam bem o seu trabalho 

\end{SingleSpacing}
\end{resumo}


