% RESUMO--------------------------------------------------------------------------------

\begin{resumo}[RESUMO]
\begin{SingleSpacing}

% Não altere esta seção do texto--------------------------------------------------------
\imprimirautorcitacao. \imprimirtitulo. \imprimirdata. \pageref {LastPage} f. \imprimirprojeto\ – \imprimirprograma, \imprimirinstituicao. \imprimirlocal, \imprimirdata.\\
%---------------------------------------------------------------------------------------

Com o avanço do poder de processamento e a miniaturização dos processadores modernos, a cada ano que se passa, vê-se cada vez mais dispositivos portáteis se tornando populares no Brasil. Esses dispositivos se utilizam de sistemas embarcados onde o consumo de recursos como bateria, processamento e memória, que são escassos, são de vital importância para que o usuário tenha uma boa experiência de uso. Atualmente, os dispositivos móveis são uma grande fonte de geração de dados. Esses dados ficam situados nas bordas da rede, sendo muitas vezes inviável tentar centralizá-los. A camada de computação em névoa surgiu como um meio de prover serviços de armazenamento e comunicação entre a computação em nuvem e os dispositivos finais (IoT - Internet of Things), podendo fornecer dados em tempo real de forma descentralizada e escalável.  Um elemento chamado servidor de mensagens fica responsável por gerenciar e distribuir a informação através de um ou mais canais de comunicação, de modo a demandar menos recursos computacionais com requisições repetidas para uma mesma informação. Este trabalho visa analisar alguns protocolos de comunicação para computação em névoa, como MQTT (Message Queuing Telemetry Transport), AMQP (Advanced Message Queuing Protocol) e STOMP (Simple Text Orientated Messaging Protocol) por meio de um servidor de mensagens RabbitMQ. O objetivo é analisar como esses protocolos se comportam nesse novo mundo desafiador de recursos escassos, que são autogerenciados por ferramentas internas do Sistema Operacional.  

\textbf{Palavras-chave}: Computação em névoa, Dispositivos Móveis, AMQP, STOMP, MQTT. 
% Escolha de 3 a 5 palavras ou termos que descrevam bem o seu trabalho 
\end{SingleSpacing}
\end{resumo}


